\documentclass{scrartcl}

\usepackage[utf8]{inputenc}

\usepackage{fixltx2e}

\usepackage{microtype}
\usepackage{amsmath}
\usepackage{amssymb}
\usepackage{mathtools}

% \usepackage{pgfplots}
% \pgfplotsset{compat=newest}

\newcommand\mytitle{Useful recurrence relations}
\newcommand\myauthor{Nico Schlömer}

\usepackage[
  pdfencoding=unicode,
  ]{hyperref}
\hypersetup{
  pdfauthor={\myauthor},
  pdftitle={\mytitle}
}

% \usepackage[T1]{fontenc}
% \usepackage{newtxtext}
% \usepackage{newtxmath}

% Okay. Don't use biblatex/biber for now. There are breaking changes in every revision,
% and we'd have to stick to the exact version that arxiv.org has, otherwise it's error
% messages like
% ```
% Package biblatex Warning: File 'main.bbl' is wrong format version
% - expected 2.8.
% ```
% \usepackage[sorting=none]{biblatex}
% \bibliography{bib}

\usepackage{bm}
\newcommand\rgb{\bm{R}}

\title{\mytitle\footnote{The LaTeX sources of this article are on \url{https://github.com/nschloe/note-useful-recurrence}}}
\author{\myauthor}

\begin{document}

\maketitle
\begin{abstract}
  Many mathematical entities, especially when working in arbitrary dimensions, are
  expressed in terms of the Gamma function, e.g., volume and surface of the hypersphere.
  These expressions are often complicated and don't provide much insight. Also, many
  expressions are unsuitable for use in a computer program. This note lists more
  suitable representations with recurrence relations for many common expressions.
\end{abstract}


\section{Volumes}

\paragraph{Volume of a unit \textit{n}-ball}

\[
  I_n
  = \frac{\sqrt{\pi}^n}{\Gamma(\frac{n}{2}+1)}
  = \begin{rcases}
    \begin{dcases}
      \frac{\pi^{\frac{n}{2}}}{(\frac{n}{2})!}&\text{if $n$ even}\\[1.2ex]
      \frac{(\frac{n+1}{2})!2^{n+1}\pi^{\frac{n-1}{2}}}{\left(n+1\right)!}&\text{if $n$ odd}
  \end{dcases}
  \end{rcases}
  = \begin{dcases}
     1&\text{if $n = 0$}\\
     2&\text{if $n = 1$}\\
     I_{n-2} \times \frac{2\pi}{n}&\text{otherwise}
  \end{dcases}
\]

From the explicit expression, it seems that $\sqrt{\pi}$ appears for odd dimensions.
Also, it is not very clear how $I_n$ behaves in higher dimension.

The recurrence makes it obvious: $I_n$ increases until about $n=5,6$, then decreases.


\paragraph{Surface of a unit \textit{n}-sphere}

\[
I_n
= \frac{n \sqrt{\pi}^n}{\Gamma(\frac{n}{2}+1)}
= \begin{rcases}\begin{dcases}
\frac{n}{(\frac{n}{2})!} \pi^{\frac{n}{2}}&\text{if $n-1$ even}\\
  \frac{n2^{n+1}(\frac{n+1}{2})!}{(n+1)!}\pi^{\frac{n-1}{2}}&\text{if $n-1$ odd}
\end{dcases}
\end{rcases}
= \begin{dcases}
  2&\text{if $n = 0$}\\
  2\pi&\text{if $n = 1$}\\
  I_{n-2} \times \frac{2\pi}{n - 2}&\text{otherwise}
\end{dcases}
\]


\paragraph{\textit{n}-dimensional Laguerre volume}

\[
  I_n = \int_{\mathbb{R}^n} \exp\left(-\sqrt{x_1^2+\dots+x_n^2}\right)
\]
\[
I_n
= \frac{2 \sqrt{\pi}^n \Gamma(n)}{\Gamma(\frac{n}{2})}
= \begin{rcases}\begin{dcases}
  \frac{2 \pi^{\frac{n}{2}} (n - 1)!}{(\frac{n}{2} - 1)!} &\text{if $n$ even}\\
  \pi^{\frac{n - 1}{2}} 2^n \left(\frac{n - 1}{2}\right)! &\text{if $n$ odd}
\end{dcases}
\end{rcases}
=\begin{dcases}
  1&\text{if $n=0$}\\
  2&\text{if $n=1$}\\
  I_{n-2} \times 2\pi(n-1)&\text{otherwise}
\end{dcases}
\]

\paragraph{\textit{n}-dimensional Hermite volume (physicists')}
\[
  I_n = \int_{\mathbb{R}^n} \exp\left(-(x_1^2+\dots+x_n^2)\right)
\]
\[
  I_n
  = \sqrt{\pi}^n
  = \begin{rcases}\begin{dcases}
     \pi^{\frac{n}{2}}&\text{if $n$ even}\\
     \sqrt{\pi} \pi^{\frac{n-1}{2}}&\text{if $n$ odd}
   \end{dcases}
   \end{rcases}
   =\begin{dcases}
     1&\text{if $n=0$}\\
     \sqrt{\pi}&\text{if $n=1$}\\
     I_{n-2} \times \pi&\text{otherwise}
   \end{dcases}
\]

\paragraph{\textit{n}-dimensional Hermite volume (probablisists')}

\[
  I_n = \frac{1}{\sqrt{2\pi}^n} \int_{\mathbb{R}^n} \exp\left(-\frac{1}{2}(x_1^2+\dots+x_n^2)\right)
\]

\[
I_n
= 1
=\begin{dcases}\begin{rcases}
  1&\text{if $n$ even}\\
  1&\text{if $n$ odd}
\end{rcases}
\end{dcases}
   =\begin{dcases}
     1&\text{if $n=0$}\\
     1&\text{if $n=1$}\\
     I_{n-2}&\text{otherwise}
   \end{dcases}
\]


% ## Monomial integrals
%
% ### Chebyshev, first kind
% See [Wikipedia](https://en.wikipedia.org/wiki/Chebyshev_polynomials).
%
% $$I_k = \int_{-1}^1 \frac{x^k}{\sqrt{1 - x^2}} dx$$
%
% $$I_k = \frac{(-1)^k + 1}{2} \frac{\sqrt{\pi} \Gamma(\frac{k+1}{2})}{\Gamma(\frac{k}{2} + 1)}$$ | $$I_k=\begin{cases}\frac{\pi k!}{2^k (\frac{k}{2})!^2}\quad&\text{if $k$ even}\\\\0\quad&\text{if $k$ odd}\end{cases}$$ | $$I_k = I_{k-2} \times \frac{k-1}{k},\quad I_0 = \pi,\quad I_1 = 0$$
% |:-----------:|:-------------:|:-----------:|
%
% ### Chebyshev, second kind
% See [Wikipedia](https://en.wikipedia.org/wiki/Chebyshev_polynomials).
% $$\int_{-1}^1 x^k \sqrt{1 - x^2} dx$$
%
% $$I_k = \frac{(-1)^k + 1}{2} \frac{\sqrt{\pi} \Gamma(\frac{k+1}{2})}{(k + 2) \Gamma(\frac{k}{2} + 1)}$$ |$$I_k=\begin{cases}\frac{\pi k!}{(k+2) 2^k (\frac{k}{2})!^2}\quad&\text{if $k$ even}\\\\0\quad&\text{if $k$ odd}\end{cases}$$| $$I_k = I_{k-2} \times \frac{k-1}{k + 2},\quad I_0 = \frac{\pi}{2},\quad I_1 = 0$$
% |:-----------:|:-------------:|:-----------:|
%
%
% ### Hermite (Physicists')
%
% See [Wikipedia](https://en.wikipedia.org/wiki/Hermite_polynomials).
% $$\int_{-\infty}^\infty x^k \exp(-x^2) dx$$
%
% $$I_k = \frac{(-1)^k + 1}{2} \Gamma\left(\frac{k+1}{2}\right)$$ | $$I_k=\begin{cases}\frac{\sqrt{\pi}k!}{2^k (\frac{k}{2})!}\quad&\text{if $k$ even}\\\\0\quad&\text{if $k$ odd}\end{cases}$$| $$I_k = I_{k-2} \times \frac{k-1}{2},\quad I_0 = \sqrt{\pi},\quad I_1 = 0$$
% |:-----------:|:-------------:|:-----------:|
%
% ### Hermite (Probabilists')
% https://en.wikipedia.org/wiki/Hermite_polynomials
% $$\frac{1}{\sqrt{2\pi}} \int_{-\infty}^\infty x^k \exp\left(-\frac{1}{2}x^2\right) dx$$
%
% $$I_k = \frac{(-1)^k + 1}{2} \frac{2^{\frac{k+1}{2}}}{\sqrt{2\pi}} \Gamma\left(\frac{k+1}{2}\right)$$ |$$I_k=\begin{cases}\frac{k!}{2^{k/2} (\frac{k}{2})!}\quad&\text{if $k$ even}\\\\0\quad&\text{if $k$ odd}\end{cases}$$| $$I_k = I_{k-2} \times (k-1),\quad I_0 = 1,\quad I_1 = 0$$
% |:-----------:|:-------------:|:-----------:|


% \printbibliography{}
\bibliography{bib}{}
\bibliographystyle{plain}

\end{document}
